\chapter{Fortran 概念}

\section{\ProgramUnit{}概念}

\subsection{\ProgramUnit{}和\ScopingUnit{}}

\ProgramUnit{}是 Fortran \Program{}的基本组件. \ProgramUnit{}是\MainProgram{}, \ExternalSubprogram{}, \Module{}, 或\Submodule{}.

\Subprogram{}是\Function{}\Subprogram{}或\Subroutine{}\Subprogram{}.

\subsection{\Program{}}

一个\Program{}应包含一个\MainProgram{}, 大于等于零个其他种类的\ProgramUnit{}, 大于等于零个\ExternalProcedure{}, 和大于等于零个以 Fortran 之外的方式\Define{}的\Entity{}.

\ExternalProcedure{}是以\ExternalSubprogram{}或 Fortran 之外的方式\Define{}的\Procedure{}.

\subsection{\Procedure{}}

\Procedure{}是\Function{}或\Subroutine{}.

\section{\Data{}概念}

\subsection{\Type{}}

\subsubsection{总述}

\Type{}是被命名的数据分类, 和其\TypeParameter{}一起决定\Value{}的集合, 表示\Value{}的语法, 和解释与操作\Value{}的\Operation{}的集合.

\Type{}是\Intrinsic{}\Type{}或\DerivedType{}.

\subsubsection{\Intrinsic{}\Type{}}

\Intrinsic{}\Type{}是\IntegerType{}, \RealType{}, \ComplexType{}, \CharacterType{}和\LogicalType{}.

所有\Intrinsic{}\Type{}都有一个\Kind{}\TypeParameter{}称作KIND, 其决定相应类型的表示方法. \CharacterType{}还有一个\Length{}\TypeParameter{}称作LEN, 其决定\CharacterString{}的\Length{}.

\subsubsection{\DerivedType{}}

\DerivedType{}可以被参数化. \DerivedType{}的\Scalar{}\Object{}是\Structure{}; \Structure{}的\Assignment{}已被\Intrinsically{}\Define{}, 但没有\Structure{}的\Intrinsic{}\Operation{}. 对每个\DerivedType{}, 都有一个\StructureConstructor{}可用于生成\Value{}. 另外, \DerivedType{}的\Object{}可用成\Procedure{}\Argument{}和\Function{}\Result{}, 并且可以出现于\Input{}/\Output{}\List{}. 如果另外的\Operation{}被\DerivedType{}所需要, 则可以由某些\Procedure{}\Define{}.

\subsection{\Data{}\Value{}}

根据\TypeParameter{}的\Value{}, 每个\Intrinsic{}\Type{}都与彼\Type{}的\Data{}可以取的\Value{}的集合联系. \DerivedType{}的\Object{}可\Assume{}的\Value{}决定于\Type{}\Define{}, \TypeParameter{}\Value{}, 和其\Component{}的\Value{}的集合.

\subsection{\Data{}\Entity{}}

\subsubsection{总述}

\Data{}\Entity{}是\Data{}\Object{}, \Expression{}的\Evaluation{}的\Result{}, 或\Function{}\Reference{}的\Execution{}的\Result{}.

一个\Data{}\Entity{}有一个\Type{}和\TypeParameter{}, 并且可能有一个\Data{}\Value{}(一个例外是\Undefined{}\Variable{}). 每个\Data{}\Entity{}都有一个\Rank{}并因此要么是一个\Scalar{}要么是一个\Array{}.

是\Function{}\Reference{}的\Execution{}的\Result{}的\Data{}\Entity{}被称为\Function{}\Result{}.

\subsubsection{\Data{}\Object{}}

\paragraph{\Data{}\Object{}分类}

\Data{}\Object{}要么是\Constant{}, 要么是\Variable{}, 要么是\Constant{}的\Subobject{}. \Named{}\Data{}\Object{}的\Type{}和\TypeParameter{}可以被\Explicitly{}或\Implicitly{}\Specify{}.

\Subobject{}是可独立于其他\Portion{}被\Reference{}的\Data{}\Object{}的一\Portion{}. 如果\Subobject{}是\Variable{}, 则还需要可独立于其他\Portion{}被\Define{}.

这些``\Portion{}''包括\Array{}的\Portion{}(\ArrayElement{}和\ArraySection{}), \CharacterString{}的\Portion{}(\Substring{}), \ComplexType{}\Object{}的\Portion{}(实部和虚部), 和\Structure{}的\Portion{}(\Component{}). \Subobject{}本身是\Data{}\Object{}, 但\Subobject{}只由\Object{}\Designator{}或\Intrinsic{}\Function{}\Reference{}. \Variable{}的\Subobject{}是\Variable{}.

\paragraph{\Variable{}}

\Variable{}可以有一个\Value{}或是\Undefined{}\Variable{}; 在\Program{}\Execution{}期间可以被\Define{}, \Redefine{}, 或重新变得\Undefined{}.

\paragraph{\Constant{}}

\Constant{}要么是\Named{}\Constant{}要么是\Literal{}\Constant{}.

\Named{}\Constant{}用 PARAMETER \Attribute{}定义.

\paragraph{\Constant{}的\Subobject{}}

\Constant{}的\Subobject{}是\Constant{}的一\Portion{}.

在\Constant{}的\Subobject{}的\Object{}\Designator{}中, 被\Reference{}的\Portion{}可能取决于某个\Variable{}的\Value{}. (示例: 在\ArraySection{}中使用\IntegerType{}\Variable{}做\Index{})

\paragraph{\Expression{}}

\Expression{}在被\Evaluate{}时生成一个\Data{}\Entity{}. \Expression{}要么表示一个\Data{}\Object{}\Reference{}要么表示一个计算, 并由\Operand{}, \Operator{}和括号组成.

\paragraph{\Function{}\Reference{}}

\Function{}\Reference{}在\Function{}于\Expression{}\Evaluation{}期间被\Execute{}时生成一个\Data{}\Entity{}. \Function{}\Result{}的\Type{}, \TypeParameter{}, 和 \Rank{}由\Function{}的接口\Interface{}决定. \Function{}\Result{}的\Value{}由\Function{}的\Execution{}决定.

\subsection{\Reference{}}

\Data{}\Object{}在其\Value{}于\Execution{}期间被请求时被\Reference{}. \Procedure{}在其被\Execute{}时被\Reference{}.

\Data{}\Object{}\Designator{}或\Procedure{}\Designator{}作为\ActualArgument{}的出现并不构成对该\Data{}\Object{}或\Procedure{}的\Reference{}, 除非这种\Reference{}对于完成\ActualArgument{}的\Specification{}是必要的.

\subsection{\Array{}}

一个\Array{}可以有最多十五减其\Corank{}个\Dimension{}, 和任意\Extent{}在其任意\Dimension{}上. 一个\Array{}的\Size{}是其所有\Element{}的总数, 等于所有\Extent{}的乘积. 一个\Array{}可以是零\Size{}的. 一个\Array{}的\Shape{}决定于其\Rank{}和其每个\Dimension{}上的\Extent{}, 并表示为一个\Element{}是其所有\Extent{}的1\Rank{}\Array{}. 所有\Named{}\Array{}必须被\Declare{}, 并且\Named{}\Array{}的\Rank{}在其\Declaration{}中被\Specify{}. 除\AssumedRankArray{}外, 一旦被\Declare{}, \Named{}\Array{}的\Rank{}就是\Constant{}.

为\Scalar{}\Object{}\Define{}的任何\Intrinsic{}\Operation{}都可以应用于\Conformable{}\Object{}. 这样的\Operation{}被\Elementally{}进行以生成一个和\Array{}的\Operand{}\Conformable{}的作为结果的\Array{}. 如果一个\Elemental{}\Operation{}是\Intrinsic{}\Prue{}\Operation{}或由一个\Prue{}\Elemental{}\Function{}实现, 那么这个\Elemental{}\Operation{}可以被同时地进行或以其他顺序进行.

一个1\Rank{}\Array{}可由\Scalar{}和其他\Array{}\Construct{}并且可以被\Reshape{}成任意允许的\Array{}\Shape{}.

\subsection{\Allocatable{}\Variable{}}

\Allocatable{}\Variable{}的\Allocation{}状态要么是\Allocated{}要么是\Unallocated{}.

一个\Unallocated{}\Variable{}不应被\Reference{}或\Define{}.

如果一个\Allocatable{}\Variable{}是一个\Array{}, 那么其\Rank{}已被\Declare{}, 但其\Bound{}在其是\Allocated{}的时候才是确定的.
