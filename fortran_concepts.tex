\chapter{Fortran 概念}

\section{\ProgramUnit{}概念}

\subsection{\ProgramUnit{}和\ScopingUnit{}}

\ProgramUnit{}是 Fortran \Program{}的基本组件. \ProgramUnit{}是\MainProgram{}, \ExternalSubprogram{}, \Module{}, 或\Submodule{}.

\Subprogram{}是\Function{}\Subprogram{}或\Subroutine{}\Subprogram{}.

\subsection{\Program{}}

一个\Program{}应包含一个\MainProgram{}, 大于等于零个其他种类的\ProgramUnit{}, 大于等于零个\ExternalProcedure{}, 和大于等于零个以 Fortran 之外的方式定义的\Entity{}.

\ExternalProcedure{}是以\ExternalSubprogram{}或 Fortran 之外的方式定义的\Procedure{}.

\subsection{\Procedure{}}

\Procedure{}是\Function{}或\Subroutine{}.

\section{数据概念}

\subsection{\Type{}}

\subsubsection{总述}

\Type{}是被命名的数据分类, 和其\TypeParameter{}一起决定\Value{}的集合, 表示\Value{}的语法, 和解释与操作\Value{}的\Operation{}的集合.

\Type{}是\Intrinsic{}\Type{}或\DerivedType{}.

\subsubsection{\Intrinsic{}\Type{}}

\Intrinsic{}\Type{}是\IntegerType{}, \RealType{}, \ComplexType{}, \CharacterType{}和\LogicalType{}.

所有\Intrinsic{}\Type{}都有一个\Kind{}\TypeParameter{}称作KIND, 其决定相应类型的表示方法. \CharacterType{}还有一个\Length{}\TypeParameter{}称作LEN, 其决定\CharacterString{}的\Length{}.

\subsubsection{\DerivedType{}}

\DerivedType{}可以被参数化. \DerivedType{}的\Scalar{}\Object{}是\Structure{}; \Structure{}的\Assignment{}已被\Intrinsically{}定义, 但没有\Structure{}的\Intrinsic{}\Operation{}. 对每个\DerivedType{}, 都有一个\StructureConstructor{}可用于生成\Value{}. 另外, \DerivedType{}的\Object{}可用成\Procedure{}\Argument{}和\Function{}\Result{}, 并且可以出现于\Input{}/\Output{}\List{}. 如果另外的\Operation{}被\DerivedType{}所需要, 则可以由某些\Procedure{}定义.
