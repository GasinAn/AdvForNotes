\chapter{\Type{}}

\section{\Type{}的特征}

\subsection{\Type{}的概念}

Fortran 提供了一种抽象的方法, 通过这种方法可以在不依赖于特定物理表现的情况下对\Data{}进行分类. 这种抽象的方法是\Type{}的概念.

一个\Type{}有一个\Name{}, 一个\Valid{}\Value{}的集合, 一个表示这些\Value{}(\Constant{})的方法, 和一个操作这些\Value{}的\Operation{}的集合.

\subsection{\Type{}分类}

\Type{}要么是\IntrinsicType{}要么是\DerivedType{}.

\DerivedType{}是由\DerivedType{}\Definition{}或\Intrinsic{}\Module{}\Define{}的, 并且应该仅当是\Accessible{}之时被使用. \IntrinsicType{}总是\Accessible{}.

\subsection{\Value{}的集合}

对每个\Type{}, 都有一个\Valid{}\Value{}的集合. \LogicalType{}的\Valid{}\Value{}的集合不是\ProcessorDependent{}. \IntegerType{}, \CharacterType{}, 和\RealType{}的\Valid{}\Value{}的集合是\ProcessorDependent{}. \ComplexType{}的\Valid{}\Value{}的集合由所有实部和虚部的\Value{}的结合的集合组成.

\subsection{\Constant{}}

表示\Value{}的语法表明\Type{}, \TypeParameter{}, 和特定的\Value{}.

是\Constant{}\Expression{}的\StructureConstructor{}表示\DerivedType{}的\Scalar{}\Constant{}\Value{}. 是\Constant{}\Expression{}的\ArrayConstructor{}表示\IntrinsicType{}或\DerivedType{}的\Constant{}\Array{}\Value{}.

\Constant{}\Value{}可以\Named{}.

对每个\IntrinsicType{}, 一个\Operation{}和相应的\Operator{}的集合以被\Intrinsically{}\Define{}. 这\Intrinsic{}集合可以由通过带 OPERATOR \Interface{}的\Function{}\Define{}的\Operation{}和\Operator{}拓充.

对\DerivedType{}, 没有\Intrinsic{}\Operation{}. \DerivedType{}上的\Operation{}可由\Program{}\Define{}.

\section{\TypeParameter{}}

如果\Type{}有\TypeParameter{}, 那么\Value{}的集合, 表示\Value{}的语法, 和\Type{}的\Value{}上的\Operation{}的集合取决于\Parameter{}的\Value{}.

\TypeParameter{}要么是\Kind{}\TypeParameter{}要么是\Length{}\TypeParameter{}. 所有\TypeParameter{}都是\IntegerType{}的. \Kind{}\TypeParameter{}参与\Generic{}\Resolution{}, 但\Length{}\TypeParameter{}不参与.

每个\IntrinsicType{}都有一个名为 KIND 的\Kind{}\TypeParameter{}. \IntrinsicType{}\CharacterType{}一个名为 LEN 的\Length{}\TypeParameter{}. \DerivedType{}可以有\TypeParameter{}.

\TypeParameter{}\Value{}可由\Type{}\Specification{}\Specify{}.

\begin{tabular}{lll}
    \tit{\TypeParameter{}\Value{}}&是&\tit{\Scalar{}\IntegerType{}\Expression{}}\\
    &或&*\\
    &或&:\\
\end{tabular}

\Kind{}\TypeParameter{}的\tit{\TypeParameter{}\Value{}}应当是\Constant{}\Expression{}.

冒号不应被用作\tit{\TypeParameter{}\Value{}}除非在有 POINTER 或 ALLOCATABLE \Attribute{}的\Entity{}的\Declaration{}中.

作为\tit{\TypeParameter{}\Value{}}的冒号\Specify{}\DeferredTypeParameter{}.

\Object{}的\DeferredTypeParameter{}的\Value{}决定于 ALLOCATE \Statement{}的成功\Execution{}, \Intrinsic{}\Assignment{}\Statement{}的\Execution{}, \Pointer{}\Assignment{}\Statement{}的\Execution{}, 或\ArgumentAssociation{}.

作为\tit{\TypeParameter{}\Value{}}的星号\Specify{}\Length{}\TypeParameter{}是\AssumedTypeParameter{}. 其被\DummyArgument{}用于从\EffectiveArgument{}中\Assume{}\TypeParameter{}\Value{}, 被 SELECT TYPE \Construct{}中的\AssociateName{}用于从相关的\Selector{}中\Assume{}\TypeParameter{}\Value{}, , 且被\CharacterType{}的\Named{}\Constant{}用于从\tit{\Constant{}\Expression{}}中\Assume{}\Character{}\Length{}.

\Kind{}\TypeParameter{}的\Value{}永远在\CompileTime{}已知.

``\Length{}\TypeParameter{}''的名称被用于非\Kind{}\TypeParameter{}的\TypeParameter{}是因为它们经常\Specify{}\Length{}. 然而, 它们也可被用于其他用途. 它们和\Kind{}\TypeParameter{}的重要区别是它们的\Value{}无需在\CompileTime{}已知并且在\Execution{}时可能变化.

\section{\Type{}, \TypeSpecifier{}, 和\Value{}}


\subsection{\Type{}与\Value{}和\Object{}的关系}

\Type{}的\Name{}充当\TypeSpecifier{}并可用于\Declare{}该\Type{}的\Object{}. \Declaration{}可以\Specify{}\Named{}\Object{}的\Type{}. \Data{}\Object{}可以\Explicitly{}或\Implicitly{}\Declare{}. \Data{}\Object{}除了其\Type{}之外还有\Attribute{}.

\Array{}由\IntrinsicType{}或\DerivedType{}的\Scalar{}\Data{}组成, 并具有与其\Element{}相同的\Type{}和\TypeParameter{}.

\Variable{}是\Data{}\Object{}. \Variable{}的\Type{}和\TypeParameter{}决定该\Variable{}可以取哪些\Value{}. \Assignment{}提供了一种改变\Variable{}的\Value{}的方法.

\Variable{}的\Type{}决定可用于操作该\Variable{}的\Operation{}.

\subsection{\TypeSpecifier{}和\Type{}\Compatibility{}}

\TypeSpecifier{}\Specify{}\Type{}和\TypeParameter{}\Value{}. 其要么是\tit{\TypeSpecifier{}}要么是\tit{\Declaration{}\TypeSpecifier{}}.

\begin{tabular}{lll}
    \tit{\TypeSpecifier{}}&是&\tit{\Intrinsic{}\TypeSpecifier{}}\\
    &或&\tit{\Derived{}\TypeSpecifier{}}\\
\end{tabular}

\Derived{}\TypeSpecifier{}不应\Specify{}\AbstractType{}.

\begin{tabular}{lll}
    \tit{\Declaration{}\TypeSpecifier{}}&是&\tit{\Intrinsic{}\TypeSpecifier{}}\\
    &或&TYPE(\tit{\Intrinsic{}\TypeSpecifier{}})\\
    &或&TYPE(\tit{\Derived{}\TypeSpecifier{}})\\
    &或&CLASS(\tit{\Derived{}\TypeSpecifier{}})\\
    &或&CLASS(*)\\
    &或&TYPE(*)\\
\end{tabular}

在\tit{\Declaration{}\TypeSpecifier{}}中, 每个不是冒号或星号的\tit{\TypeParameter{}\Value{}}都应该\Specification{}\Expression{}.

在使用 CLASS \Keyword{}的\tit{\Declaration{}\TypeSpecifier{}}中, \tit{\Derived{}\TypeSpecifier{}}应该\Specify{}一个\ExtensibleType{}.

TYPE(\tit{\Derived{}\TypeSpecifier{}}) 不应\Specify{}\AbstractType{}.

在 TYPE(\tit{\Intrinsic{}\TypeSpecifier{}}) 中, \tit{\Intrinsic{}\TypeSpecifier{}}不能以逗号结尾.

用 CLASS \Keyword{}\Declare{}的\Entity{}应该是一个\DummyArgument{}或具有ALLOCATABLE 或 POINTER \Attribute{}.

注: TYPE \TypeSpecifier{}和CLASS \TypeSpecifier{}的区别在于TYPE \TypeSpecifier{}只代表某一\Type{}而CLASS \TypeSpecifier{}代表某一\Type{}及其所有子\Type{}.

\subsection{TYPE \TypeSpecifier{}}

TYPE \TypeSpecifier{}用于\Declare{}\AssumedType{}\Entity{}, 或\IntrinsicType{}或\DerivedType{}\Entity{}.

\Type{}\Declaration{}\Statement{}中的 TYPE \TypeSpecifier{}中的\tit{\Derived{}\TypeSpecifier{}}应当\Specify{}先前被\Define{}的\DerivedType{}. 如果\Data{}\Entity{}是\Function{}\Result{}, 则只要\DerivedType{}是在\FunctionBody{}中被\Define{}的或在\FunctionBody{}中通过\UseAssociation{}或\HostAssociation{}是\Accessible{}\DerivedType{}就可在 FUNCTION \Statement{} 中被\Specify{}. 如果\DerivedType{}在FUNCTION \Statement{}中被\Specify{}并在\FunctionBody{}中被\Define{}, 那么这就好像\Function{}\Result{}紧接着被\Specify{}的\DerivedType{}的\tit{\DerivedType{}\Definition{}}用彼\DerivedType{}被\Declare{}.

\subsection{CLASS \TypeSpecifier{}}

CLASS \TypeSpecifier{}用于\Declare{}\Polymorphic{}\Entity{}。\Polymorphic{}\Entity{}是在\Program{}\Execution{}期间能够具有不同\DynamicType{}的\Data{}\Entity{}。\DynamicType{}是\Data{}\Entity{}在\Program{}\Execution{}期间的特定点的\Type{}.

\Type{}\Declaration{}\Statement{}中的 CLASS \TypeSpecifier{}中的\tit{\Derived{}\TypeSpecifier{}}应当\Specify{}先前被\Define{}的\DerivedType{}. 如果\Data{}\Entity{}是\Function{}\Result{}, 则只要\DerivedType{}是在\FunctionBody{}中被\Define{}的或在\FunctionBody{}中通过\UseAssociation{}或\HostAssociation{}是\Accessible{}\DerivedType{}就可在 FUNCTION \Statement{} 中被\Specify{}. 如果\DerivedType{}在FUNCTION \Statement{}中被\Specify{}并在\FunctionBody{}中被\Define{}, 那么这就好像\Function{}\Result{}紧接着彼\DerivedType{}的\tit{\DerivedType{}\Definition{}}用彼\DerivedType{}被\Declare{}.

如果 CLASS \TypeSpecifier{}包含\Type{}\Name{},那么\Polymorphic{}\Entity{}的\DeclaredType{}为被\Specify{}的\Type{}。\DeclaredType{}是\Data{}\Entity{}被要么\Explicitly{}要么\Implicitly{}\Declare{}具有的\Type{}.

\Nonpolymorphic{}\Entity{}仅和相同\DeclaredType{}的\Entity{}有\Type{}\Compatibility{}. 不是\UnlimitedPolymorphic{}\Entity{}的\Polymorphic{}\Entity{}和相同\DeclaredType{}的或其\Extension{}的\Entity{}有\Type{}\Compatibility{}. 如果\Entity{}与某\Type{}的\Entity{}有\Type{}\Compatibility{}那么其与该\Type{}有\Type{}\Compatibility{}.
