\chapter{\Type{}}

\section{\Type{}的特征}

\subsection{\Type{}的概念}

Fortran 提供了一种抽象的方法, 通过这种方法可以在不依赖于特定物理表现的情况下对\Data{}进行分类. 这种抽象的方法是\Type{}的概念.

一个\Type{}有一个\Name{}, 一个\Valid{}\Value{}的集合, 一个表示这些\Value{}(\Constant{})的方法, 和一个操作这些\Value{}的\Operation{}的集合.

\subsection{\Type{}分类}

\Type{}要么是\IntrinsicType{}要么是\DerivedType{}.

\DerivedType{}是由\DerivedType{}\Definition{}或\Intrinsic{}\Module{}\Define{}的, 并且应该仅当是\Accessible{}之时被使用. \IntrinsicType{}总是\Accessible{}.

\subsection{\Value{}的集合}

对每个\Type{}, 都有一个\Valid{}\Value{}的集合. \LogicalType{}的\Valid{}\Value{}的集合不是\ProcessorDependent{}. \IntegerType{}, \CharacterType{}, 和\RealType{}的\Valid{}\Value{}的集合是\ProcessorDependent{}. \ComplexType{}的\Valid{}\Value{}的集合由所有实部和虚部的\Value{}的结合的集合组成.

\subsection{\Constant{}}

表示\Value{}的语法表明\Type{}, \TypeParameter{}, 和特定的\Value{}.

是\Constant{}\Expression{}的\StructureConstructor{}表示\DerivedType{}的\Scalar{}\Constant{}\Value{}. 是\Constant{}\Expression{}的\ArrayConstructor{}表示\IntrinsicType{}或\DerivedType{}的\Constant{}\Array{}\Value{}.

\Constant{}\Value{}可以\Named{}.

对每个\IntrinsicType{}, 一个\Operation{}和相应的\Operator{}的集合以被\Intrinsically{}\Define{}. 这\Intrinsic{}集合可以由通过带 OPERATOR \Interface{}的\Function{}\Define{}的\Operation{}和\Operator{}拓充.

对\DerivedType{}, 没有\Intrinsic{}\Operation{}. \DerivedType{}上的\Operation{}可由\Program{}\Define{}.

\section{\TypeParameter{}}

如果\Type{}有\TypeParameter{}, 那么\Value{}的集合, 表示\Value{}的语法, 和\Type{}的\Value{}上的\Operation{}的集合取决于\Parameter{}的\Value{}.

\TypeParameter{}要么是\Kind{}\TypeParameter{}要么是\Length{}\TypeParameter{}. 所有\TypeParameter{}都是\IntegerType{}的. \Kind{}\TypeParameter{}参与\Generic{}\Resolution{}, 但\Length{}\TypeParameter{}不参与.

每个\IntrinsicType{}都有一个名为 KIND 的\Kind{}\TypeParameter{}. \IntrinsicType{}\CharacterType{}一个名为 LEN 的\Length{}\TypeParameter{}. \DerivedType{}可以有\TypeParameter{}.

\TypeParameter{}\Value{}可由\Type{}\Specification{}\Specify{}.

\begin{tabular}{ccc}
    \it{\TypeParameter{}\Value{}}&是&\it{\Scalar{}\IntegerType{}\Expression{}}\\
    &或&*\\
    &或&:\\
\end{tabular}

\Kind{}\TypeParameter{}的\it{\TypeParameter{}\Value{}}应当是\Constant{}\Expression{}.

冒号不应被用作\it{\TypeParameter{}\Value{}}除非在有 POINTER 或 ALLOCATABLE \Attribute{}的\Entity{}的\Declaration{}中.

作为\it{\TypeParameter{}\Value{}}的冒号\Specify{}\DeferredTypeParameter{}.

\Object{}的\DeferredTypeParameter{}的\Value{}决定于 ALLOCATE \Statement{}的成功\Execution{}, \Intrinsic{}\Assignment{}\Statement{}的\Execution{}, \Pointer{}\Assignment{}\Statement{}的\Execution{}, 或\ArgumentAssociation{}.

作为\it{\TypeParameter{}\Value{}}的星号\Specify{}\Length{}\TypeParameter{}是\AssumedTypeParameter{}. 其被\DummyArgument{}用于从\EffectiveArgument{}中\Assume{}\TypeParameter{}\Value{}, 被 SELECT TYPE \Construct{}中的\AssociateName{}用于从相关的\Selector{}中\Assume{}\TypeParameter{}\Value{}, , 且被\CharacterType{}的\Named{}\Constant{}用于从\it{\Constant{}\Expression{}}中\Assume{}\Character{}\Length{}.

\Kind{}\TypeParameter{}的\Value{}永远在\CompileTime{}已知.

``\Length{}\TypeParameter{}''的名称被用于非\Kind{}\TypeParameter{}的\TypeParameter{}是因为它们经常\Specify{}\Length{}. 然而, 它们也可被用于其他用途. 它们和\Kind{}\TypeParameter{}的重要区别是它们的\Value{}无需在\CompileTime{}已知并且在\Execution{}时可能变化.
